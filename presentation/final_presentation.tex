% ===
%
% Official LaTeX beamer template of 
% Chair for AI Methodology (AIM)
% RWTH Aachen University, Aachen, Germany
%
% Author: Jakob Bossek (bossek@aim.rwth-aachen.de)
% Based on earlier presentation slide templates by 
% Jakob Bossek and Holger H. Hoos.
% 
% AIM website: https://aim.rwth-aachen.de/
%
% ===

% HINT: add option aspectratio=169 for 16:9 aspect ratio
% HINT: add 'handout' to activate handout mode (without overlays for printing)
\documentclass[t,english]{beamer}

% include package import and macro defintions
% tables
\RequirePackage{booktabs}
\RequirePackage{colortbl}
\RequirePackage{multicol}
\RequirePackage{multirow}
\RequirePackage{xspace}
\RequirePackage[final]{pdfpages}

% \smiley{} and \frowney{}
\RequirePackage{wasysym}

% quotation
\RequirePackage{csquotes}

\input{includes/rgb}
\input{includes/macros/general}
\input{includes/macros/commenting}

% add (multiple) bibliography sources for biblatex
\addbibresource{bib.bib}

% NOTE: the predefined options are best practices of AIM!
%       Do not change for seminar talks!
% Set authorinfo={0,1} to (de)activate short author and short title in footer
% Set progress={0,1} to switch between x and x/total display of pages in the bottom right corner
% Set outline={0,1} to (de)activate outline slides at the beginning of each section
\usetheme[authorinfo=1, progress=0, outline=0]{AIM}

% metadata
\title[Rapid Time Series Datasets Library]{Efficient AI with Rust Lab \newline Rapid Time Series Datasets Library
\newline
RWTH Aachen University}
\subtitle{Group 1}
\author[Aali \& Kaufmann \& Braun]{Marius Kaufmann\inst{1} \and Amir Ali Aali\inst{2} \and Kilian Fin Braun\inst{1}}
\institute{
\inst{1}Masters of Computer Science\\
\inst{2}Masters of Data Science\\
}
\date{\small\today}

% optional
\titlegraphic{
  \includegraphics[width=2cm]{figures/rwth-logo.png}
}

%%% NOTE: do not remove the following marker. We use it for automatic compilation of multiple versions
%%% of the slides (e.g., handout, presentation with overlays) via compile.py
%%%pythonmarker

\begin{document}

\begin{frame}[plain]
\titlepage
\end{frame}

\addtocounter{framenumber}{-1}

%%% ================================= Marius's Part =================================
\begin{frame}
  \frametitle{Marius's Part}
    
\end{frame}

%%% ================================= Amir's part =================================
\begin{frame}
  \frametitle{Downsampling I}
    \textbf{Goal:} Reduce the number of data points in a time series dataset.

    \begin{block}<2->{Benefits:}
      \begin{itemize}
        \item<2-> Reduces memory usage
        \item<3-> Speeds up processing time
      \end{itemize}
    \end{block}

    \begin{block}<4->{Neccessary parameter when downsampling:}
      \begin{itemize}
        \item<4-> Downsampling factor: How many data points to skip
      \end{itemize}
    \end{block}

    \begin{block}<5->{Example:}
      \begin{itemize}
        \item<5-> Downsampling factor of 2: Every second data point is kept as shown in Figure \ref{fig:downsampling}
      \end{itemize}
    \end{block}

\end{frame}

\begin{frame}
  \frametitle{Downsampling II}
    \begin{figure}[H]
        \includegraphics[width=0.9\textwidth]{figures/downsampling/downsampling.png}
        \caption{Downsampling example with a factor of 2}
        \label{fig:downsampling}
    \end{figure}

\end{frame}

\begin{frame}
  \frametitle{Downsampling III}
    \begin{block}<1->{How it works:}
      \begin{itemize}
        \item<1-> The downsampling function takes a time series dataset and a downsampling factor as input.
        \item<2-> It iterates over the dataset and keeps every n-th data point, where n is the downsampling factor.
      \end{itemize}
    \end{block}

    \begin{block}<3->{Bottleneck of passing the data by reference:}
      \begin{itemize}
        \item<3-> Not possible. A copy is needed.
        \item <4-> Creating view only possible on contiguous data.
        \item <5-> Downsampling does not yield a contiguous data structure.
      \end{itemize}
    \end{block}
\end{frame}

\begin{frame}
  \frametitle{Splitting I}
    \textbf{Goal:} Split a time series dataset into three parts: training, validation, and test.

    \begin{block}<2->{Different splitting strategies:}
      \begin{itemize}
        \item<2-> Random split (Classification Data)
        \item<3-> In-Order split (Classification Data)
        \item<4-> Temporal split (Forecasting Data)
      \end{itemize}
    \end{block}

    \begin{block}<5->{Neccessary parameter when splitting:}
      \begin{itemize}
        \item<5-> Training set ratio
        \item<6-> Validation set ratio
        \item<7-> Test set ratio
      \end{itemize}
    \end{block}
\end{frame}

\begin{frame}
  \frametitle{Splitting II (Random Split - Classification Data)}
    \begin{block}<1->{How it works:}
      \begin{enumerate}
        \item<1-> Validate the proportions of train, validation, and test sets.
        \item <2-> Shuffle the dataset randomly.
        \item <3-> Compute the split offsets based on the proportions.
        \item <4-> Split the instances into three sets.
        \item <5-> Return the three sets as separate datasets.
      \end{enumerate}
    \end{block}
\end{frame}

\begin{frame}
  \frametitle{Splitting III (Random Split - Classification Data)}
    \begin{figure}[H]
        \includegraphics[width=1\textwidth]{figures/splitting/random_split.png}
        \caption{Random split example}
        \label{fig:random_split}
    \end{figure}
\end{frame}

\begin{frame}
  \frametitle{Splitting IV (In-Order Split - Classification Data)}
    Works very similar to the random split, but it \textbf{doesn't shuffle} the dataset anymore.

    \begin{figure}[H]
        \includegraphics[width=1\textwidth]{figures/splitting/in_order_split.png}
        \caption{In-Order split example}
        \label{fig:in_order_split}
    \end{figure}
\end{frame}

\begin{frame}
  \frametitle{Splitting V (Temporal Split - Forecasting Data)}
    Similar to the in-order split, but this time we are dealing with forecasting data, which in most cases is only one instance and we split over \textbf{timestamps} and not isntances anymore.

    \begin{figure}[H]
        \includegraphics[width=1\textwidth]{figures/splitting/temporal_split.png}
        \caption{Temporal split example}
        \label{fig:temporal_split}
    \end{figure}
\end{frame}

\begin{frame}
  \frametitle{Standardization}
    \textbf{Goal:} Transform each feature in a time series dataset to have a mean of 0 and a standard deviation of 1.

    \begin{block}<2->{How it works}
      \begin{itemize}
        \item<2-> Compute the mean and standard deviation for each feature column in the dataset.
        \item<3-> Through a for-loop iterate over each feature and apply the standardization formula:
        \begin{equation}
              x' = \frac{x - \text{mean}}{\text{std}}
        \end{equation}
        \item<4-> Apply the same mean and standard deviation to the validation and test sets.
      \end{itemize}
    \end{block}
\end{frame}

\begin{frame}
  \frametitle{Min-Max Normalization}
    \textbf{Goal:} Transform each feature in a time series dataset to a range between 0 and 1.

    \begin{block}<2->{How it works}
      \begin{itemize}
        \item<2-> Compute the minimum and maximum for each feature in the dataset.
        \item<3-> Through a for-loop iterate over each feature and apply the min-max normalization formula:
        \begin{equation}
              x' = \frac{x - \text{min}}{\text{max} - \text{min}}
        \end{equation}
        \item<4-> Apply the same min and max to the validation and test sets.
      \end{itemize}
    \end{block}
\end{frame}



%%% ================================= Kilian's part =================================
\begin{frame}
  \frametitle{Kilian's Part}
    
\end{frame}

\end{document}
