% ===
%
% Official LaTeX beamer template of 
% Chair for AI Methodology (AIM)
% RWTH Aachen University, Aachen, Germany
%
% Author: Jakob Bossek (bossek@aim.rwth-aachen.de)
% Based on earlier presentation slide templates by 
% Jakob Bossek and Holger H. Hoos.
% 
% AIM website: https://aim.rwth-aachen.de/
%
% ===

% HINT: add option aspectratio=169 for 16:9 aspect ratio
% HINT: add 'handout' to activate handout mode (without overlays for printing)
\documentclass[t,english]{beamer}
\usepackage{multimedia}

% include package import and macro defintions
% tables
\RequirePackage{booktabs}
\RequirePackage{colortbl}
\RequirePackage{multicol}
\RequirePackage{multirow}
\RequirePackage{xspace}
\RequirePackage[final]{pdfpages}

% \smiley{} and \frowney{}
\RequirePackage{wasysym}

% quotation
\RequirePackage{csquotes}

\input{includes/rgb}
\input{includes/macros/general}
\input{includes/macros/commenting}
\definecolor{classColor}{RGB}{0,0,139}    % Dark Blue for classes
\definecolor{methodColor}{RGB}{63,92,161} % Darker Turquoise for methods
\definecolor{typeColor}{RGB}{139,0,0}  % Dark Red for types
\definecolor{commentColor}{RGB}{34,139,34} % ForestGreen for comments
\definecolor{variableColor}{RGB}{0,139,139} % Slightly darker Turquoise for variables
\definecolor{libraryColor}{RGB}{0,0,0} % Black for libraries

% Define Rust as a custom language
\lstdefinelanguage{Rust}{
    keywords={abstract, alignof, as, become, box, break, const, continue, crate, do, else, enum, extern, false, final, fn, for, if, impl, in, let, loop, macro, match, mod, move, mut, offsetof, override, priv, proc, pub, pure, ref, return, Self, self, sizeof, static, struct, super, trait, true, type, typeof, unsafe, unsized, use, virtual, where, while, yield},
    morecomment=[l]{//},
    morecomment=[s]{/*}{*/},
    morestring=[b]",
    morestring=[b]',
}

\lstdefinestyle{python}{
    language=Python,
    basicstyle=\ttfamily\small,
    keywordstyle=\color{blue},
    commentstyle=\color{commentColor},
    stringstyle=\color{red},
    morekeywords={self, None, True, False, import, from, as, def, class, return, if, elif, else, for, while, break, continue, pass, try, except, finally, raise},
    showstringspaces=false,
    % Custom highlighting for classes, variables and methods
    classoffset=4, % Offset for variable styling
    morekeywords={forecasting_data_set, forecasting_data_set_res, classification_data_set, classification_data_set_res, dataset, dataset_type, past_window, future_horizon, stride, labels, batch_size, num_workers, downsampling_rate, impute_strategy, splitting_strategy, splitting_ratios}, % List of variables
    keywordstyle=\color{variableColor},
    classoffset=3, % Separate class styling
    morekeywords={ForecastingDataSet, ClassificationDataSet, DatasetType, ImputeStrategy, SplittingStrategy}, % List of classes
    keywordstyle=\color{classColor},
    classoffset=2, % Reset for methods
    morekeywords={impute, downsample, split, normalize, standardize, collect, __init__}, % List of methods
    keywordstyle=\color{methodColor},
    classoffset=1, % Reset for types
    morekeywords={np, ndarray, LeaveNaN, Temporal},
    keywordstyle=\color{typeColor},
    showstringspaces=false,
}

\lstdefinestyle{rust}{
    language=Rust,
    basicstyle=\ttfamily\small,
    keywordstyle=\color{blue},
    commentstyle=\color{commentColor},
    stringstyle=\color{red},
    showstringspaces=false,
    % Custom highlighting for classes, variables and methods
    classoffset=3, % Offset for variable styling
    morekeywords={train_view, val_view, test_view, self, _py, train_data, val_data, test_data, data, train_split_index, val_split_index}, % List of variables
    keywordstyle=\color{variableColor},
    classoffset=2, % Separate class styling
    morekeywords={ArrayBase, Dim, PyResult, DataMut, Elem, S, Ok, Python, SplittingStrategy}, % List of classes
    keywordstyle=\color{classColor},
    classoffset=1, % Reset for methods
    morekeywords={impute, downsample, split, normalize, standardize, collect, as_mut, unwrap, get_split_views_mut, check_arrays_set}, % List of methods
    keywordstyle=\color{methodColor},
    showstringspaces=false,
}

% generics
\newcommand{\rustCode}[3]{\texttt{\textcolor{#2}{#1}#3}\xspace}
\newcommand{\rustFunc}[2]{\rustCode{#1}{methodColor}{(\textcolor{variableColor}{#2})}}
\newcommand{\rustNoArgsFunc}[1]{\rustFunc{#1}{}}
\newcommand{\rustClass}[2]{\rustCode{#1}{classColor}{#2}}
\newcommand{\rustGenClass}[1]{\rustClass{#1}{<...>}}
\newcommand{\rustNoArgsClass}[1]{\rustClass{#1}{}}
\newcommand{\rustVar}[1]{\rustCode{#1}{variableColor}{}}
\newcommand{\rustLibrary}[1]{\rustCode{#1}{libraryColor}{}}

% functions
\newcommand{\impute}{\rustNoArgsFunc{impute}}
\newcommand{\downsample}{\rustNoArgsFunc{downsample}}
\newcommand{\downsampleFac}{\rustFunc{downsample}{factor}}
\newcommand{\splitShort}{\rustNoArgsFunc{split}}
\newcommand{\splitFore}{\rustFunc{split}{train\_part, val\_part, test\_part}}
\newcommand{\normalize}{\rustNoArgsFunc{normalize}}
\newcommand{\standardize}{\rustNoArgsFunc{standardize}}
\newcommand{\collect}{\rustNoArgsFunc{collect}}
\newcommand{\collectFore}{\rustFunc{collect}{past\_window, future\_horizon, stride}}
\newcommand{\setup}{\rustNoArgsFunc{setup}}

% classes
\newcommand{\arraybase}{\rustGenClass{ArrayBase}}
\newcommand{\arrayrust}{\rustGenClass{Array}}
\newcommand{\arrayview}{\rustGenClass{ArrayView}}
\newcommand{\py}{\rustGenClass{Py}}
\newcommand{\classificationDataSet}{\rustNoArgsClass{ClassificationDataSet}}
\newcommand{\forecastingDataSet}{\rustNoArgsClass{ForecastingDataSet}}
\newcommand{\lightningModule}{\rustNoArgsClass{LightningModule}}
\newcommand{\lightningDM}{\rustNoArgsClass{LightningDataModule}}
\newcommand{\rustDM}{\rustNoArgsClass{RustDataModule}}
\newcommand{\dataLoader}{\rustNoArgsClass{DataLoader}}

% variables
\newcommand{\pastWindow}{\rustVar{past\_window}}
\newcommand{\futureHorizon}{\rustVar{future\_horizon}}
\newcommand{\stride}{\rustVar{stride}}
\newcommand{\splitStrategy}{\rustVar{split\_strategy}}

% libraries
\newcommand{\numpy}{\rustLibrary{numpy}}
\newcommand{\torch}{\rustLibrary{torch}}


% add (multiple) bibliography sources for biblatex
\addbibresource{bib.bib}

% NOTE: the predefined options are best practices of AIM!
%       Do not change for seminar talks!
% Set authorinfo={0,1} to (de)activate short author and short title in footer
% Set progress={0,1} to switch between x and x/total display of pages in the bottom right corner
% Set outline={0,1} to (de)activate outline slides at the beginning of each section
\usetheme[authorinfo=1, progress=0, outline=0]{AIM}

% metadata
\title[Rapid Time Series Datasets Library]{Efficient AI with Rust Lab \newline Rapid Time Series Datasets Library
\newline
RWTH Aachen University}
\subtitle{Group 1}
\author[Aali \& Kaufmann \& Braun]{Marius Kaufmann\inst{1} \and Amir Ali Aali\inst{2} \and Kilian Fin Braun\inst{1}}
\institute{
\inst{1}Masters of Computer Science\\
\inst{2}Masters of Data Science\\
}
\date{\small\today}

% optional
\titlegraphic{
  \includegraphics[width=2cm]{figures/rwth-logo.png}
}

%%% NOTE: do not remove the following marker. We use it for automatic compilation of multiple versions
%%% of the slides (e.g., handout, presentation with overlays) via compile.py
%%%pythonmarker

\begin{document}

\begin{frame}[plain]
  \titlepage
\end{frame}

\addtocounter{framenumber}{-1}

%%% ================================= Marius's Part =================================
\begin{frame}
  \frametitle{Overview}

  \begin{block}{Goal}
    \begin{itemize}
      \item<1-> Preprocessing of time series datasets
      \item<2-> Python package implemented in Rust
      \item<3-> Passing data by reference
            \begin{itemize}
              \item <4-> Using \numpy crate
            \end{itemize}
    \end{itemize}
  \end{block}

  \begin{block}<5->{Scope}
    \begin{itemize}
      \item<5-> Two types of datasets
            \begin{itemize}
              \item<6-> \forecastingDataSet
              \item<7-> \classificationDataSet
            \end{itemize}
      \item<8-> Functionality
            \begin{itemize}
              \item <8-> \impute
              \item <9-> \downsample
              \item <10-> \splitShort
              \item <11-> \normalize / \standardize
            \end{itemize}
    \end{itemize}
  \end{block}
\end{frame}

\begin{frame}
  \frametitle{Data Input Format}

  \begin{columns}
    \begin{column}{0.6\textwidth}
      \begin{block}{Input 3D numpy array:}
        \begin{itemize}
          \item<2-> \textbf{First dimension:} Instances
          \item<3-> \textbf{Second dimension:} Timesteps
          \item<4-> \textbf{Third dimension:} Features
        \end{itemize}
      \end{block}

      \begin{block}<7->{In practice}
        \begin{itemize}
          \item<7-> Forecasting datasets:
                \begin{itemize}
                  \item<7-> One instance
                \end{itemize}
          \item<8-> Classification datasets:
                \begin{itemize}
                  \item<8-> Multiple instances
                \end{itemize}
        \end{itemize}
      \end{block}
    \end{column}

    \begin{column}{0.4\textwidth}
      \only<5> {
        \begin{figure}[H]
          \includegraphics[width=1\textwidth]{figures/1d.png}
        \end{figure}
      }
      \only<6-> {
        \begin{figure}[H]
          \includegraphics[width=1\textwidth]{figures/3d.png}
        \end{figure}
      }
    \end{column}
  \end{columns}

\end{frame}

% SPLITTING HERE?

\begin{frame}
  \frametitle{Performance considerations}

  \begin{block}{Copying}
    \begin{itemize}
      \item <1-> Copying data is expensive
      \item <2-> Avoid unnecessary copies
      \item <3-> Copy only when absolutely necessary
            \begin{itemize}
              \item <4-> Only once
            \end{itemize}
    \end{itemize}
  \end{block}

  \begin{block}<5->{When to copy?}
    \begin{itemize}
      \item<6-> For \forecastingDataSet:
            \begin{itemize}
              \item<7-> Windowed format in final step
              \item<8-> Copying unavoidable
            \end{itemize}
      \item<9-> For \classificationDataSet:
            \begin{itemize}
              \item<10-> Random splitting strategy offered
              \item<11-> Copying unavoidable
            \end{itemize}
    \end{itemize}

  \end{block}
\end{frame}

\begin{frame}
  \frametitle{Data-flow}

  \begin{figure}[H]
    \centering
    \scalebox{0.58}{
      \begin{tikzpicture}[
          node distance=1.5cm,
          every node/.style={align=center},
          process/.style={rectangle, draw, fill=blue!20, minimum width=2.5cm, minimum height=0.8cm},
          data/.style={rectangle, draw, fill=green!20, minimum width=2.5cm, minimum height=0.8cm},
          decision/.style={diamond, draw, fill=yellow!20, minimum width=2cm, minimum height=0.8cm},
          arrow/.style={->, thick},
          dashedarrow/.style={->, thick, dashed},
          copy/.style={rectangle, draw, fill=red!20, minimum width=2.5cm, minimum height=0.8cm}    ]

        % Title
        \node[above] at (0, 8) {\textbf{Forecasting Dataset Data-Flow}};
        \node[above] at (6, 8) {\textbf{Classification Dataset Data-Flow}};

        % Forecasting flow (left side)
        \node[data, visible on=<1->] (f_data) at (0, 7) {Original Data\\(NumPy Array)};
        \node[process, visible on=<2->] (f_constructor) at (0, 5.5) {Constructor\\(Store Reference)};
        \node[process, visible on=<3->] (f_impute) at (0, 4) {\impute\\(Optional)};
        \node[copy, visible on=<4->] (f_downsample) at (0, 2.5) {\downsample\\(Optional)};
        \node[process, visible on=<5->] (f_split) at (0, 1) {\splitShort\\(Compute Split Indices)};
        \node[process, visible on=<6->] (f_normalize) at (0, -0.5) {\normalize/\standardize\\(Array Views, Optional)};
        \node[copy, visible on=<7->] (f_collect) at (0, -2) {\collect\\(Copy + Sliding Windows)};
        \node[data, visible on=<8->] (f_result) at (0, -3.5) {Train/Val/Test\\(Sliding Windows)};

        % Classification flow (right side)
        \node[data, visible on=<1->] (c_data) at (6, 7) {Original Data\\(NumPy Array)};
        \node[process, visible on=<2->] (c_constructor) at (6, 5.5) {Constructor\\(Store Reference)};
        \node[process, visible on=<3->] (c_impute) at (6, 4) {\impute\\(Optional)};
        \node[copy, visible on=<4->] (c_downsample) at (6, 2.5) {\downsample\\(Optional)};
        \node[copy, visible on=<5->] (c_split) at (6, 1) {\splitShort\\(Split + Copy)};
        \node[process, visible on=<6->] (c_normalize) at (6, -0.5) {\normalize/\standardize\\(Owned Arrays, Optional)};
        \node[process, visible on=<7->] (c_collect) at (6, -2) {\collect\\(Return Arrays)};
        \node[data, visible on=<8->] (c_result) at (6, -3.5) {Train/Val/Test\\(Original Format)};

        % Arrows for forecasting flow
        \draw[arrow, visible on=<2->] (f_data) -- (f_constructor);
        \draw[arrow, visible on=<3->] (f_constructor) -- (f_impute);
        \draw[arrow, visible on=<4->] (f_impute) -- (f_downsample);
        \draw[arrow, visible on=<5->] (f_downsample) -- (f_split);
        \draw[arrow, visible on=<6->] (f_split) -- (f_normalize);
        \draw[arrow, visible on=<7->] (f_normalize) -- (f_collect);
        \draw[arrow, visible on=<8->] (f_collect) -- (f_result);

        % Arrows for classification flow
        \draw[arrow, visible on=<2->] (c_data) -- (c_constructor);
        \draw[arrow, visible on=<3->] (c_constructor) -- (c_impute);
        \draw[arrow, visible on=<4->] (c_impute) -- (c_downsample);
        \draw[arrow, visible on=<5->] (c_downsample) -- (c_split);
        \draw[arrow, visible on=<6->] (c_split) -- (c_normalize);
        \draw[arrow, visible on=<7->] (c_normalize) -- (c_collect);
        \draw[arrow, visible on=<8->] (c_collect) -- (c_result);

        % Legend
        \node[data, minimum width=3cm] at (12, 7) {Data Storage};
        \node[process, minimum width=3cm] at (12, 6.2) {Processing Step};
        \node[copy, minimum width=3cm] at (12, 5.4) {Data Copying};

        % Side annotations
        \node[left, visible on=<6->] at (-3, 1) {\footnotesize Only indices\\computed};
        \node[left, visible on=<7->] at (-3, -0.5) {\footnotesize Works on\\array views};
        \node[left, visible on=<8->] at (-3, -2) {\footnotesize Single copy\\operation};

        \node[right, visible on=<6->] at (9, 1) {\footnotesize Actual data\\splitting (copying)};
        \node[right, visible on=<7->] at (9, -0.5) {\footnotesize Works on\\owned arrays};
        \node[right, visible on=<8->] at (9, -2) {\footnotesize No additional\\copying};

      \end{tikzpicture}
    }
    \label{fig:data-flow-comparison}
  \end{figure}
\end{frame}

\begin{frame}[fragile]
  \frametitle{Pipeline Design}

  \begin{columns}
    \begin{column}{0.5\textwidth}
      \forecastingDataSet

      \begin{lstlisting}[style=python,mathescape=false,basicstyle=\ttfamily\scriptsize]
# Create instance
fore = ForecastingDataSet(
  data, 0.7, 0.2, 0.1
)

# call the pipeline methods
fore.impute(
  ImputeStrategy.Median
)
fore.downsample(2)
fore.split()


fore.normalize()
fore.standardize()

# collect the results
fore_res = fore.collect(3, 1, 1)
            \end{lstlisting}
    \end{column}

    \pause

    \begin{column}{0.5\textwidth}
      \classificationDataSet

      \begin{lstlisting}[style=python,mathescape=false,basicstyle=\ttfamily\scriptsize]
# create instance
clas = ClassificationDataSet(
  data, labels, 0.7, 0.2, 0.1
)

# call the pipeline methods
clas.impute(
  ImputeStrategy.Median
)
clas.downsample(2)
clas.split(
  SplittingStrategy.Random
)
clas.normalize()
clas.standardize()

# collect the results
clas_res = clas.collect()
            \end{lstlisting}
    \end{column}
  \end{columns}
\end{frame}

%%% ================================= Amir's part =================================
\begin{frame}
  \frametitle{Downsampling I}
  \textbf{Goal:} Reduce the number of data points in a time series dataset.

  \begin{block}<2->{Benefits:}
    \begin{itemize}
      \item<2-> Reduces memory usage
      \item<3-> Speeds up processing time
    \end{itemize}
  \end{block}

  \begin{block}<4->{Neccessary parameter when downsampling:}
    \begin{itemize}
      \item<4-> Downsampling factor: How many data points to skip
    \end{itemize}
  \end{block}

  \begin{block}<5->{Example:}
    \begin{itemize}
      \item<5-> Downsampling factor of 2: Every second data point is kept
    \end{itemize}
  \end{block}

\end{frame}

\begin{frame}
  \frametitle{Downsampling II}
  \begin{figure}[H]
    \includegraphics[width=0.9\textwidth]{figures/downsampling/downsampling.png}
    \caption{Downsampling example with a factor of 2}
    \label{fig:downsampling}
  \end{figure}

\end{frame}

\begin{frame}
  \frametitle{Downsampling III}
  \begin{block}<1->{How it works:}
    \begin{itemize}
      \item<1-> The downsampling function takes a time series dataset and a downsampling factor as input.
      \item<2-> It iterates over the dataset and keeps every n-th data point, where n is the downsampling factor.
    \end{itemize}
  \end{block}

  \begin{block}<3->{Bottleneck of passing the data by reference:}
    \begin{itemize}
      \item<3-> Not possible. A copy is needed.
      \item <4-> Creating view only possible on contiguous data.
      \item <5-> Downsampling does not yield a contiguous data structure.
    \end{itemize}
  \end{block}
\end{frame}

\begin{frame}
  \frametitle{Splitting I}
  \textbf{Goal:} Split a time series dataset into three parts: training, validation, and test.

  \begin{block}<2->{Different splitting strategies:}
    \begin{itemize}
      \item<2-> Random split (Classification Data)
      \item<3-> In-Order split (Classification Data)
      \item<4-> Temporal split (Forecasting Data)
    \end{itemize}
  \end{block}

  \begin{block}<5->{Neccessary parameter when splitting:}
    \begin{itemize}
      \item<5-> Training set ratio
      \item<6-> Validation set ratio
      \item<7-> Test set ratio
    \end{itemize}
  \end{block}
\end{frame}

\begin{frame}
  \frametitle{Splitting II (Random Split - Classification Data)}
  \begin{block}<1->{How it works:}
    \begin{enumerate}
      \item<1-> Validate the proportions of train, validation, and test sets.
      \item <2-> Shuffle the instances of the dataset randomly.
      \item <3-> Compute the split offsets based on the proportions.
      \item <4-> Split the instances into three sets.
      \item <5-> Return the three sets as separate datasets.
    \end{enumerate}
  \end{block}
\end{frame}

\begin{frame}
  \frametitle{Splitting III (Random Split - Classification Data)}
  \begin{figure}[H]
    \includegraphics[width=1\textwidth]{figures/splitting/random_split.png}
    \caption{Random split example}
    \label{fig:random_split}
  \end{figure}
\end{frame}

\begin{frame}
  \frametitle{Splitting IV (In-Order Split - Classification Data)}
  Works very similar to the random split, but it \textbf{doesn't shuffle} the dataset anymore.

  \begin{figure}[H]
    \includegraphics[width=1\textwidth]{figures/splitting/in_order_split.png}
    \caption{In-Order split example}
    \label{fig:in_order_split}
  \end{figure}
\end{frame}

\begin{frame}
  \frametitle{Splitting V (Temporal Split - Forecasting Data)}
  Similar to the in-order split, but this time we are dealing with forecasting data, which in most cases is only one instance and we split over \textbf{timesteps} and not instances anymore.

  \begin{figure}[H]
    \includegraphics[width=1\textwidth]{figures/splitting/temporal_split.png}
    \caption{Temporal split example}
    \label{fig:temporal_split}
  \end{figure}
\end{frame}

\begin{frame}
  \frametitle{Standardization}
  \textbf{Goal:} Transform each feature in a time series dataset to have a \textbf{mean} of \textbf{0} and a \textbf{standard deviation} of \textbf{1}.

  \begin{block}<2->{How it works}
    \begin{itemize}
      \item<2-> Compute the mean and standard deviation for each feature column in the \textbf{training} dataset.
      \item<3-> Through a for-loop iterate over each feature and apply the standardization formula:
            \begin{equation}
              x' = \frac{x - \text{mean}}{\text{std}}
            \end{equation}
      \item<4-> Apply the same mean and standard deviation to the \textbf{validation} and \textbf{test} sets.
    \end{itemize}
  \end{block}
\end{frame}

\begin{frame}
  \frametitle{Min-Max Normalization}
  \textbf{Goal:} Transform each feature in a time series dataset to a range \textbf{between 0} and \textbf{1}.

  \begin{block}<2->{How it works}
    \begin{itemize}
      \item<2-> Compute the minimum and maximum for each feature in the \textbf{training} dataset.
      \item<3-> Through a for-loop iterate over each feature and apply the min-max normalization formula:
            \begin{equation}
              x' = \frac{x - \text{min}}{\text{max} - \text{min}}
            \end{equation}
      \item<4-> Apply the same min and max to the \textbf{validation} and \textbf{test} sets.
    \end{itemize}
  \end{block}
\end{frame}

\begin{frame}
  \frametitle{Unit Tests I}
  \textbf{Goal:} Ensure the correctness of the implemented methods.
  
    \begin{block}<2->{Bottleneck:}
        \begin{itemize}
        \item <2-> Rust code is tightly integrated with PyO3.
        \item <3-> PyO3 is not compatible with the standard Rust testing framework.
        \item <4-> PyO3 is a Rust crate that allows Rust code to be called from Python.
        \item <5-> PyO3 provides a way to write Python bindings for Rust code.
        \end{itemize}
    \end{block}

    \begin{block}<6->{Solution:}
        \begin{itemize}
        \item <6-> Use the PyO3 testing framework.
        \item <7-> Mimic the Python API in Rust.
        \item <8-> Write unit tests in Rust that can be called from Python.
        \item <9-> Use the PyO3 testing framework to run the tests.
        \end{itemize}
    \end{block}

\end{frame}

\begin{frame}
  \frametitle{Unit Tests II}
  \textbf{Example:} Testing the \impute method.

  \begin{block}<1->{How it works:}
    \begin{itemize}
      \item<1-> Create a simpe numpy array with missing values.
      \item<2-> Call the \impute method with a specific strategy.
      \item<3-> Check if the missing values are filled correctly.
    \end{itemize}
  \end{block}

    \begin{block}<4->{Coverage:}
        \begin{itemize}
        \item <4-> The unit tests cover most of the implemented methods.
        \item <5-> Since tests are not native Rust tests, we couldn't use the standard Rust coverage tools.
        \item <6-> We used the PyO3 testing framework to run the tests and check the coverage.
        \item <7-> The coverage is not as detailed as with the standard Rust testing framework, but it is sufficient for our needs.
        \end{itemize}
    \end{block}

\end{frame}

\begin{frame}
    \frametitle{Unit Tests III}
    \begin{block}<1->{How we calculated the coverage:}
        \begin{itemize}
        \item<1-> We used the PyO3 testing framework to run the tests.
        \item<2-> Counted the number of all methods.
        \item <3-> Counted the number of methods that were called during the tests.
        \item <4-> Calculated the coverage as a percentage.
        \end{itemize}
    \end{block}

    \begin{block}<5->{Results:}
        \begin{itemize}
        \item<5-> Number of all methods: 47
        \item <6-> Number of methods called during tests: 40
        \item <7-> Coverage: 85.1\%
        \end{itemize}
    \end{block}
\end{frame}


%%% ================================= Kilian's part =================================
\begin{frame}
  \frametitle{Kilian's Part}

\end{frame}

\end{document}
