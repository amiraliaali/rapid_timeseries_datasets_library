% ===
%
% Official LaTeX beamer template of 
% Chair for AI Methodology (AIM)
% RWTH Aachen University, Aachen, Germany
%
% Author: Jakob Bossek (bossek@aim.rwth-aachen.de)
% Based on earlier presentation slide templates by 
% Jakob Bossek and Holger H. Hoos.
% 
% AIM website: https://aim.rwth-aachen.de/
%
% ===

% HINT: add option aspectratio=169 for 16:9 aspect ratio
% HINT: add 'handout' to activate handout mode (without overlays for printing)
\documentclass[t,english]{beamer}

% include package import and macro defintions
% tables
\RequirePackage{booktabs}
\RequirePackage{colortbl}
\RequirePackage{multicol}
\RequirePackage{multirow}
\RequirePackage{xspace}
\RequirePackage[final]{pdfpages}

% \smiley{} and \frowney{}
\RequirePackage{wasysym}

% quotation
\RequirePackage{csquotes}

\input{includes/rgb}
\input{includes/macros/general}
\input{includes/macros/commenting}

% add (multiple) bibliography sources for biblatex
\addbibresource{bib.bib}

% NOTE: the predefined options are best practices of AIM!
%       Do not change for seminar talks!
% Set authorinfo={0,1} to (de)activate short author and short title in footer
% Set progress={0,1} to switch between x and x/total display of pages in the bottom right corner
% Set outline={0,1} to (de)activate outline slides at the beginning of each section
\usetheme[authorinfo=1, progress=0, outline=0]{AIM}

% metadata
\title[Rapid Time Series Datasets Library]{Efficient AI with Rust Lab \newline Rapid Time Series Datasets Library
\newline
RWTH Aachen University}
\subtitle{Group 1}
\author[Aali \& Kaufmann \& Braun]{Marius Kaufmann\inst{1} \and Amir Ali Aali\inst{2} \and Kilian Fin Braun\inst{1}}
\institute{
\inst{1}Masters of Computer Science\\
\inst{2}Masters of Data Science\\
}
\date{\small\today}

% optional
\titlegraphic{
  \includegraphics[width=2cm]{figures/rwth-logo.png}
}

%%% NOTE: do not remove the following marker. We use it for automatic compilation of multiple versions
%%% of the slides (e.g., handout, presentation with overlays) via compile.py
%%%pythonmarker

\begin{document}

\begin{frame}[plain]
\titlepage
\end{frame}

\addtocounter{framenumber}{-1}

%%% ================================= Marius's Part =================================
\begin{frame}
  \frametitle{Marius's Part}
    
\end{frame}

%%% ================================= Amir's part =================================
\begin{frame}
  \frametitle{Data Abstraction I}
  In time series datasets, we often have to deal with mainly two types of data:
  \begin{itemize}[]
    \item \textbf{Forecasting Data:}
    \begin{itemize}
      \item Contains only floating point values
      \item Used for predicting future values
      \item Example: Stock prices, weather data
    \end{itemize}
    \item \textbf{Classification Data:}
    \begin{itemize}
      \item Contains a mix of floating point and categorical values
      \item Used for classifying time series data into categories
      \item Example: Medical data, sensor data
    \end{itemize}
  \end{itemize}
  
\end{frame}

\begin{frame}
  \frametitle{Data Abstraction II}
  We require categorical columns to be provided as either one-hot or label-encoded values.

  This enables us to save both datasets in a unified way, which is a table of floating point values.

  \begin{figure}[H]
    \includegraphics[width=0.9\textwidth]{figures/example_categorical.png}
  \end{figure}

\end{frame}

\begin{frame}
  \frametitle{Data Point Representation}
  For our current implementation, we defined a function \textit{.get(index)} that returns a data point at the given index.

  In each of the two dataset types, we have a different representation of the data point.
  \begin{itemize}[]
    \item \textbf{Forecasting Data Point:}
    \begin{itemize}
      \item \textbf{ID:} A unique identifier for the data point
      \item \textbf{Past:} A vector of floating point values representing past observations
      \item \textbf{Future:} A vector of floating point values representing future observations
    \end{itemize}
    \item \textbf{Classification Data Point:}
    \begin{itemize}
      \item \textbf{ID:} A unique identifier for the data point
      \item \textbf{Features:} A vector of floating point values representing the features of the data point
      \item \textbf{Label:} A vector of floating point values representing the label of the data point
    \end{itemize}
  \end{itemize}

\end{frame}

\begin{frame}
  \frametitle{Splitting Strategies}
  As one of the main features of our library, we provide different splitting strategies for the datasets.
  \begin{itemize}[]
    \item \textbf{Random Split:}
    \begin{itemize}
      \item Randomly splits the dataset into training and test sets
      \item Can be used only for classification data
    \end{itemize}
    \item \textbf{Temporal Split:}
    \begin{itemize}
      \item Splits the dataset ordered by time
      \item Can be used for both forecasting and classification data
    \end{itemize}
  \end{itemize}

\end{frame}

%%% ================================= Kilian's part =================================
\begin{frame}
  \frametitle{Kilian's Part}
    
\end{frame}

\end{document}
